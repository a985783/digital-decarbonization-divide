% !TEX program = xelatex
\documentclass[11pt]{article}
\usepackage[margin=1in]{geometry}
\usepackage{graphicx}
\usepackage{booktabs}
\usepackage{amsmath}
\usepackage{amssymb}
\usepackage{hyperref}
\usepackage{float}
\usepackage{natbib}
\usepackage{setspace}

% 中文支持 - 使用 Mac 系统字体
\usepackage{xeCJK}
\setCJKmainfont{STKaiti}
\setCJKsansfont{Heiti SC}
\setCJKmonofont{STKaiti}

\title{数字化脱碳机制:基于高维面板数据与双重机器学习的实证研究}
\author{崔庆松\\独立研究者}
\date{2026年1月21日}

\begin{document}
\maketitle

\begin{abstract}
本文利用高维双重机器学习(DML)框架研究数字经济(以ICT服务出口为代理变量)与CO$_2$排放之间的关系。为克服数据稀疏性问题,我们采用链式方程多重插补(MICE)方法保持样本量($N=960$)。我们实施了具有双向固定效应和国家组交叉验证的防泄漏面板DML设计。研究结果表明,在排除能源控制变量的基准规格中,ICT与碳排放之间存在统计显著的负相关关系($\theta \approx -0.028$,$p < 0.01$)。然而,当加入能源使用作为协变量时,该估计值衰减并失去显著性。高敏感性表明基准关联主要由能源相关因素驱动。辅助回归显示ICT对总体能源使用没有显著因果效应,表明观察到的衰减可能反映的是对能源强度混杂因素的敏感性,而非简单的中介渠道。
\end{abstract}

\noindent\textbf{JEL分类号:} C14, C23, O33, Q56\\
\textbf{关键词:} 双重机器学习、高维数据、MICE插补、能源效率、数字经济、CO$_2$排放

%==============================================================================
\section{引言}
%==============================================================================

数字经济推动环境可持续发展的潜力是一个备受争议的话题。虽然数字化提供了去物质化和提高效率的途径,但它也带来了来自数据中心、网络基础设施和电子设备的日益增长的能源足迹\citep{Lange2020}。此前的实证研究结果参差不齐,往往受到小样本量、遗漏变量偏差和线性函数形式假设的限制\citep{Salahuddin2016}。

理解ICT与碳排放的关系对气候政策至关重要。如果数字化确实能减少排放,政策制定者应加速数字基础设施投资。相反,如果ICT扩张通过回弹效应增加排放,则需要不同的减排策略。文献中从强烈负效应到正效应的矛盾证据凸显了严格因果识别的必要性。

\subsection{文献综述}

ICT与碳排放之间的关系已通过多种理论视角进行研究。\textit{去物质化假说}认为,数字技术替代了实物产品和出行,减少了物质吞吐量和排放\citep{Berkhout2000}。相反,\textit{回弹效应假说}认为,ICT带来的效率提升被消费增加所抵消\citep{Gossart2015}。

实证证据仍然不一致。\citet{Salahuddin2016}发现OECD国家的ICT与排放呈正相关,将其归因于数字基础设施的能源需求。\citet{Danish2018}报告新兴经济体呈现负效应,表明数字化使这些国家能够跨越式发展到更清洁的技术。近期的元分析强调,结果对样本选择、变量定义和估计方法高度敏感\citep{Lange2020}。

一个关键的方法论缺口是依赖于假设同质效应的线性面板模型,这些模型难以处理高维混杂问题。传统的固定效应估计无法灵活控制经济发展、制度质量与排放之间复杂的非线性关系。由\citet{Chernozhukov2018}提出的双重机器学习(DML)通过使用机器学习建模干扰函数同时保持有效的因果推断来解决这一局限性。

\subsection{研究贡献}

本研究通过应用严格的数据工程和因果推断技术推进了文献发展。我们解决了三个关键挑战:

\begin{enumerate}
    \item \textbf{防泄漏面板DML实现:}我们提供了一个可复制的蓝图,结合了双向固定效应、国家组交叉验证(GroupKFold)和聚类稳健推断,用于跨国ML应用。这解决了标准交叉验证泄漏国家特定信息的常见陷阱。
    
    \item \textbf{重新评估ICT-排放证据:}在严格的识别和缺失数据处理下,估计效应对能源控制变量的模型设定高度敏感。这种敏感性对解释先前研究发现具有重要意义。
    
    \item \textbf{机制一致的敏感性证据:}通过比较有无能源使用控制变量(可能是后处理变量)的防泄漏DML规格,我们记录了ICT系数对能源相关协变量的强敏感性。
\end{enumerate}

\subsection{核心发现}

我们发现,在基准规格中,ICT服务出口对人均CO$_2$排放具有\textbf{统计显著的负效应}($\theta = -0.028$,$p < 0.01$)。然而,该估计对能源相关控制变量的纳入敏感。这种模式表明,数字经济与较低排放的关联通过能源相关渠道(如能源强度或能源结构)运作,而非作为直接的外生冲击。

本文其余部分组织如下:第2节描述数据来源和样本构建;第3节介绍方法论,包括DML框架和识别假设;第4节报告实证结果;第5节讨论发现及其含义;第6节总结。

%==============================================================================
\section{数据与样本构建}
%==============================================================================

\subsection{数据来源}

我们从世界银行的两个数据库获取数据:世界发展指标(WDI)和全球治理指标(WGI)。WDI提供经济、社会和环境变量,而WGI提供六个维度的制度质量指标。数据通过批量下载获取(2026年1月),以确保可复制性。

\subsection{样本选择}

我们的初始样本包括40个主要经济体(20个OECD国家,20个非OECD国家),观测期为2000年至2023年。在应用数据质量筛选后:
\begin{itemize}
    \item 两个国家(越南和西班牙)因结果/处理变量覆盖不足被排除。
    \item 描述性样本包含38个经济体,960个国家-年份观测值。
    \item 尼日利亚因缺少控制变量在某些规格中被剔除。
\end{itemize}

\subsection{变量定义}

表~\ref{tab:variables}总结了主要变量。处理变量是ICT服务出口占服务出口总额的百分比,反映一个国家在数字服务领域的专业化程度。结果变量是人均CO$_2$排放(公吨)。

\begin{table}[H]
\centering
\caption{变量定义}
\label{tab:variables}
\begin{tabular}{lll}
\toprule
变量 & 定义 & 来源 \\
\midrule
\multicolumn{3}{l}{\textit{核心变量}} \\
CO$_2$排放 & 人均(公吨) & WDI \\
ICT服务出口 & 占服务出口百分比 & WDI \\
\midrule
\multicolumn{3}{l}{\textit{控制变量(共60个)}} \\
人均GDP & 2015年不变美元 & WDI \\
能源使用 & 人均石油当量公斤 & WDI \\
可再生能源 & 占总能源消费百分比 & WDI \\
贸易开放度 & (出口+进口)/GDP & WDI \\
城市人口 & 占总人口百分比 & WDI \\
互联网用户 & 占人口百分比 & WDI \\
制度质量 & 6个WGI维度 & WGI \\
\bottomrule
\end{tabular}
\end{table}

为处理高维特征空间($P = 60$),我们纳入了八个主题领域的变量:制度质量(6个WGI指标)、人口与社会因素、基础设施与数字连接、金融深度、宏观经济结构、能源与环境以及创新指标。

\subsection{描述性统计}

表~\ref{tab:desc}展示了分析样本中主要变量的统计摘要。

\begin{table}[H]
\centering
\caption{描述性统计($N = 960$)}
\label{tab:desc}
\begin{tabular}{lrrrrr}
\toprule
变量 & 观测数 & 均值 & 标准差 & 最小值 & 最大值 \\
\midrule
CO$_2$排放(公吨/人) & 960 & 5.82 & 4.67 & 0.12 & 22.45 \\
ICT服务出口(\%) & 960 & 8.94 & 9.52 & 0.31 & 58.23 \\
人均GDP(美元) & 960 & 24,518 & 22,134 & 412 & 108,542 \\
能源使用(千克石油当量/人) & 960 & 3,245 & 2,089 & 142 & 8,356 \\
可再生能源(\%) & 960 & 18.42 & 16.78 & 0.02 & 78.45 \\
互联网用户(\%) & 960 & 52.34 & 28.67 & 0.08 & 99.72 \\
\bottomrule
\end{tabular}
\end{table}

\noindent\textit{注:}统计数据在MICE插补后计算。CO$_2$单位为人均公吨。

\subsection{缺失数据处理}

缺失数据在跨国面板中普遍存在。我们采用\textbf{链式方程多重插补(MICE)}处理控制变量,而非列表删除(这将使样本减少40\%以上)。关键要点:
\begin{itemize}
    \item 我们\textit{从不}插补结果变量($Y$)或处理变量($T$),对这些核心变量执行完整案例分析。
    \item 插补\textit{仅在每个训练折中}进行,以防止信息泄漏。
    \item 最终推断通过Rubin规则合并5个插补数据集的估计。
\end{itemize}

%==============================================================================
\section{研究方法}
%==============================================================================

\subsection{面板DML框架}

我们使用部分线性模型估计因果参数$\theta$:
\begin{equation}
Y_{it} = \theta T_{it} + g(W_{it}) + \alpha_i + \gamma_t + \epsilon_{it}
\label{eq:plm}
\end{equation}
其中$Y_{it}$是CO$_2$排放,$T_{it}$是ICT出口,$W_{it}$是高维控制变量向量,$\alpha_i$捕获国家固定效应,$\gamma_t$捕获年份固定效应。

DML方法\citep{Chernozhukov2018}分两个阶段进行:
\begin{enumerate}
    \item \textbf{干扰估计:}使用机器学习估计$\hat{g}(W) = \mathbb{E}[Y|W]$和$\hat{m}(W) = \mathbb{E}[T|W]$。
    \item \textbf{正交化回归:}将残差化结果$\tilde{Y} = Y - \hat{g}(W)$对残差化处理变量$\tilde{T} = T - \hat{m}(W)$进行回归。
\end{enumerate}

关键洞见是$\theta$具有"双重稳健性":只要$\hat{g}$或$\hat{m}$之一正确设定,估计就保持一致性;即使两者都以低于参数化的速率估计,推断仍然有效。

\subsection{识别假设}

我们的因果解释依赖于标准假设:

\begin{enumerate}
    \item \textbf{条件不混杂:}$Y(t) \perp T | W, \alpha_i, \gamma_t$对所有处理水平$t$成立。
    \item \textbf{重叠性:}$0 < P(T = t | W, \alpha_i, \gamma_t) < 1$对支撑集中所有$t$成立。
    \item \textbf{无干扰:}一国的ICT出口不影响另一国的排放(SUTVA)。
\end{enumerate}

双向固定效应吸收了时不变的国家异质性和共同时间冲击。高维控制变量$W$旨在捕获剩余混杂因素。我们仍需谨慎,因为能源使用可能是中介变量而非混杂变量;因此,我们展示了包含和排除能源控制变量的规格。

\subsection{干扰学习器}

对于干扰函数$g(W)$和$m(W)$,我们使用\textbf{梯度提升回归器(XGBoost)},参数设置如下:
\begin{itemize}
    \item 最大深度:4
    \item 学习率:0.05
    \item 估计器数量:200
    \item 正则化:L2惩罚($\lambda = 1$)
\end{itemize}

这些设置在灵活性和正则化之间取得平衡,防止高维环境下的过拟合。

\subsection{交叉验证策略}

标准K折交叉验证在面板数据中可能泄漏信息——当同一国家的观测同时出现在训练集和测试集时。我们实施\textbf{GroupKFold}($K = 5$),按国家分组。这确保:
\begin{itemize}
    \item 没有国家同时出现在训练和预测折中。
    \item 干扰模型仅学习跨国模式。
    \item 国家特定固定效应通过组内变换吸收。
\end{itemize}

\subsection{聚类推断}

标准误在国家层面聚类,以考虑国家内部的相关性。我们应用小样本校正$G/(G-1)$,其中$G$是聚类数。置信区间使用$G-1$自由度的$t$分布。

\subsection{变量选择}

为降低维数,我们使用\textbf{时间序列交叉验证的LassoCV}进行初步变量选择。图~\ref{fig:lasso}展示了Lasso系数路径,说明随着正则化减弱,模型如何逐步选择预测变量。

\begin{figure}[H]
\centering
\includegraphics[width=0.85\linewidth]{results/figures/lasso_path_v3.png}
\caption{Lasso系数路径。随着正则化参数$\alpha$减小(从左到右),模型选择关键预测变量,包括能源使用、可再生能源和固定电话。}
\label{fig:lasso}
\end{figure}

%==============================================================================
\section{实证结果}
%==============================================================================

\subsection{主要因果估计}

表~\ref{tab:main}展示了四种规格下的DML估计。关键比较是包含与排除能源使用控制变量的规格之间的对比。

\begin{table}[H]
\centering
\caption{DML因果估计:ICT对CO$_2$排放的影响}
\label{tab:main}
\begin{tabular}{lcccccc}
\toprule
规格 & $\theta$ & 标准误 & $p$值 & 95\%置信区间 & $N$ \\
\midrule
(1) Lasso选择 & $-0.007$ & 0.006 & 0.248 & [$-0.019$, $0.005$] & 960 \\
(2) Lasso(无能源) & $-0.027^{***}$ & 0.009 & 0.003 & [$-0.044$, $-0.009$] & 960 \\
(3) 完整高维 & $-0.013$ & 0.009 & 0.145 & [$-0.031$, $0.005$] & 960 \\
(4) 完整高维(无能源) & $-0.028^{***}$ & 0.010 & 0.005 & [$-0.048$, $-0.009$] & 960 \\
\bottomrule
\end{tabular}
\end{table}
\noindent\textit{注:}$^{***}p<0.01$,$^{**}p<0.05$,$^{*}p<0.1$。标准误按国家聚类。双向固定效应。干扰学习器:XGBoost。交叉验证:按国家分组的GroupKFold($K=5$)。

\subsection{结果解读}

结果揭示了一个显著的模式:

\begin{enumerate}
    \item \textbf{基准效应(无能源控制):}在规格(2)和(4)中,ICT出口对排放显示出统计显著的负效应。ICT服务出口每增加1个百分点,人均CO$_2$排放减少约27-28公斤。
    
    \item \textbf{含能源控制:}当纳入能源使用[规格(1)和(3)]时,ICT系数衰减约50-75\%并失去统计显著性。
    
    \item \textbf{敏感性解读:}这种衰减表明能源相关因素对解读ICT系数至关重要。两种解释与此模式一致:
    \begin{itemize}
        \item \textit{中介效应:}ICT\textit{通过}降低能源强度来减少排放。
        \item \textit{混杂效应:}能源相关因素混杂ICT与排放的关系。
    \end{itemize}
\end{enumerate}

\subsection{机制分析}

为探究机制,我们以能源使用为结果变量进行辅助DML回归:

\begin{table}[H]
\centering
\caption{机制分析:ICT对能源使用的影响}
\label{tab:mechanism}
\begin{tabular}{lcccc}
\toprule
路径 & $\theta$ & 标准误 & $p$值 & 解释 \\
\midrule
ICT $\to$ 能源使用 & $-1.20$ & 2.60 & 0.645 & 不显著 \\
能源使用 $\to$ CO$_2$ & $0.0011^{***}$ & 0.0001 & $<0.001$ & 高度显著 \\
\bottomrule
\end{tabular}
\end{table}

ICT对总体能源使用的零效应表明,衰减模式\textit{并非}由ICT减少总能源消费的简单中介渠道驱动。相反,敏感性可能反映:
\begin{itemize}
    \item 能源\textit{强度}(单位能源的排放)而非总能源水平。
    \item 与能源密集型产业结构相关的遗漏变量偏差。
\end{itemize}

\subsection{探索性非线性证据}

图~\ref{fig:shap_sum}展示了SHAP汇总图,显示辅助XGBoost模型的特征重要性。能源使用和可再生能源份额排名最高,与物理直觉一致。ICT相关变量(互联网用户、ICT出口)也显示出显著重要性。

\begin{figure}[H]
\centering
\includegraphics[width=0.85\linewidth]{results/figures/shap_summary_v3.png}
\caption{SHAP汇总图:辅助预测模型的特征重要性排名。}
\label{fig:shap_sum}
\end{figure}

图~\ref{fig:shap}展示了ICT出口的SHAP依赖图。该图揭示了一个潜在的非线性模式:当ICT服务出口超过服务出口总额的约\textbf{6\%}后,负相关关系似乎加强。

\begin{figure}[H]
\centering
\includegraphics[width=0.85\linewidth]{results/figures/shap_dependence_v3.png}
\caption{SHAP依赖图:转折区域约在6\%附近的非线性模式。}
\label{fig:shap}
\end{figure}

我们强调,这种阈值模式是\textbf{探索性的},来自预测(而非因果)模型。应将其视为假设生成而非已确认的因果阈值。

%==============================================================================
\section{讨论}
%==============================================================================

\subsection{结果解读}

我们的发现为数字化与脱碳的持续争论做出了贡献。统计显著的基准效应($\theta \approx -0.028$)表明,在控制大量混杂因素后,ICT专业化程度较高的国家往往人均排放较低。然而,对能源控制变量的敏感性提醒我们不应做出直接的因果解释。

两种情景与我们的结果一致:

\begin{enumerate}
    \item \textbf{能源效率渠道:}ICT发展使生产过程更加高效,减少单位经济产出的排放。这将表现为显著的基准效应,在控制能源强度后衰减。
    
    \item \textbf{结构性混杂:}专注于ICT服务的国家可能具有较轻的产业结构(较少制造业,较多服务业),这独立地产生较低的排放。ICT系数将捕获这种结构差异而非因果效应。
\end{enumerate}

我们的机制分析提供了混合证据:ICT对总能源使用的零效应不支持简单中介,但不排除对能源\textit{构成}或\textit{强度}的影响。

\subsection{与先前文献的比较}

我们的结果有助于调和文献中的矛盾发现:
\begin{itemize}
    \item 报告大负效应的研究可能遗漏了能源相关控制变量。
    \item 报告零效应或正效应的研究可能纳入了吸收间接渠道的控制变量。
    \item 探索性的约6\%阈值表明存在线性模型无法检测的异质效应。
\end{itemize}

\subsection{局限性}

若干局限性需要谨慎对待:

\begin{enumerate}
    \item \textbf{测量:}ICT服务出口仅捕获数字化的一个维度。替代指标(ICT资本存量、宽带普及率、数据中心容量)可能产生不同结果。
    
    \item \textbf{MICE假设:}多重插补假设数据随机缺失(MAR)。违反此假设可能使估计产生偏差。
    
    \item \textbf{外部有效性:}我们40个主要经济体的样本可能无法推广到较小或发展中国家。
    
    \item \textbf{阈值解读:}约6\%的转折点是探索性的,需要结构模型(如阈值DML、因果森林)的确认。
\end{enumerate}

\subsection{政策启示}

尽管存在这些局限性,我们的发现提供了初步的政策指导:
\begin{itemize}
    \item 数字化转型政策可能有助于脱碳目标,但机制可能通过结构转型运作。
    \item 潜在的阈值效应表明,数字基础设施可能需要达到"临界规模"才能实现减排。
    \item 政策制定者应同时监测ICT发展和能源构成,以厘清机制。
\end{itemize}

%==============================================================================
\section{结论}
%==============================================================================

本文使用防泄漏面板DML设计、高维控制变量和MICE插补协变量,重新检验了ICT与CO$_2$的关系。我们发现,在排除能源使用控制变量的基准规格中,ICT服务出口与人均CO$_2$排放的显著降低相关($\theta \approx -0.028$,$p < 0.01$)。然而,当加入能源使用作为控制变量时,估计值衰减并失去统计显著性。

这种衰减表明能源相关因素对解读ICT系数至关重要,但仅凭加入/剔除控制变量的比较无法区分中介效应和混杂效应。辅助机制分析显示ICT对总体能源使用没有显著效应,表明敏感性反映的是能源强度混杂因素而非简单中介。

辅助预测模型的探索性SHAP模式表明,在低ICT强度(约6\%)附近预测关系可能发生变化,但这应被视为假设生成而非因果阈值。

\subsection{未来研究方向}

未来研究应当:
\begin{enumerate}
    \item 使用正式的阈值DML或因果森林方法检验阈值假设。
    \item 检验不同国家类型(OECD与非OECD、制造业与服务业经济体)的异质效应。
    \item 纳入能源强度的直接测量作为潜在中介变量。
    \item 将分析扩展到行业层面数据,以分离特定行业的数字效应。
\end{enumerate}

%==============================================================================
\section*{数据与代码可用性}
%==============================================================================

所有代码和处理后的数据均包含在本仓库中:
\begin{itemize}
    \item \textbf{数据工程:}\texttt{scripts/solve\_wdi\_v4\_expanded\_zip.py}、\texttt{scripts/impute\_mice.py}
    \item \textbf{变量选择:}\texttt{scripts/lasso\_selection.py}
    \item \textbf{因果推断:}\texttt{scripts/dml\_causal\_v2.py}
    \item \textbf{机制分析:}\texttt{scripts/xgboost\_shap\_v3.py}
    \item \textbf{结果:}\texttt{results/dml\_results\_v3.csv}
\end{itemize}

%==============================================================================
\bibliographystyle{apalike}
\begin{thebibliography}{99}

\bibitem[Berkhout and Hertin(2000)]{Berkhout2000}
Berkhout, F. and Hertin, J. (2000).
\newblock De-materialising the economy or rematerialisation? The case of information and communication technologies.
\newblock \textit{The Environmental Impact of Prosperous Societies}, 4, 12--26.

\bibitem[Chernozhukov et al.(2018)]{Chernozhukov2018}
Chernozhukov, V., Chetverikov, D., Demirer, M., Duflo, E., Hansen, C., Newey, W., and Robins, J. (2018).
\newblock Double/debiased machine learning for treatment and structural parameters.
\newblock \textit{The Econometrics Journal}, 21(1), C1--C68.

\bibitem[Danish et al.(2018)]{Danish2018}
Danish, Zhang, B., Wang, B., and Wang, Z. (2018).
\newblock Role of renewable energy and non-renewable energy consumption on EKC: Evidence from Pakistan.
\newblock \textit{Journal of Cleaner Production}, 156, 855--864.

\bibitem[Gossart(2015)]{Gossart2015}
Gossart, C. (2015).
\newblock Rebound effects and ICT: A review of the literature.
\newblock In \textit{ICT Innovations for Sustainability} (pp. 435--448). Springer.

\bibitem[Lange et al.(2020)]{Lange2020}
Lange, S., Pohl, J., and Santarius, T. (2020).
\newblock Digitalization and energy consumption. Does ICT reduce energy demand?
\newblock \textit{Ecological Economics}, 176, 106760.

\bibitem[Lundberg and Lee(2017)]{Lundberg2017}
Lundberg, S. M. and Lee, S.-I. (2017).
\newblock A unified approach to interpreting model predictions.
\newblock In \textit{Advances in Neural Information Processing Systems 30} (pp. 4765--4774).

\bibitem[Salahuddin and Alam(2016)]{Salahuddin2016}
Salahuddin, M. and Alam, K. (2016).
\newblock Information and Communication Technology, electricity consumption and economic growth in OECD countries: A panel data analysis.
\newblock \textit{International Journal of Electrical Power \& Energy Systems}, 76, 185--193.

\bibitem[World Bank(2026)]{WorldBank2026}
World Bank. (2026).
\newblock \textit{World Development Indicators}.
\newblock Washington, D.C.: The World Bank.

\end{thebibliography}

%==============================================================================
\appendix
\section*{附录:变量选择细节}

Lasso路径分析(图~\ref{fig:lasso})从60个候选变量中识别关键预测变量。随着$\alpha$减小保持非零的变量被认为是重要预测变量。最终选择的变量集包括:能源使用、可再生能源、人均GDP、城市人口、互联网用户和移动通信订阅。

\end{document}
