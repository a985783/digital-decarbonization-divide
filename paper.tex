\documentclass[12pt]{article}
\usepackage[margin=1in]{geometry}
\usepackage{graphicx}
\usepackage{booktabs}
\usepackage{amsmath}
\usepackage{amssymb}
\usepackage{hyperref}
\usepackage{float}
\usepackage{natbib}
\usepackage{setspace}
\usepackage{titlesec}

% Hyperlink setup
\hypersetup{
    colorlinks=true,
    linkcolor=blue,
    filecolor=magenta,      
    urlcolor=cyan,
    citecolor=blue,
}

\begin{document}

%==============================================================================
% Title Page
%==============================================================================
\begin{titlepage}
    \centering
    \vspace*{1cm}
    
    {\Large \textbf{The Digital Decarbonization Divide: Asymmetric Effects of ICT on CO$_2$ Emissions Across Socio-economic Capacity}}
    
    \vspace{1.5cm}
    
    \textbf{Qingsong Cui} \\
    Independent Researcher \\
    \href{mailto:a985783827@gmail.com}{a985783827@gmail.com}
    
    \vspace{1.5cm}
    
    \today
    
    \vspace{2cm}
    
    \begin{abstract}
        \noindent Using a Causal Forest framework (2,000 trees) with rigorous overfitting controls (honest splitting, country-clustered cross-fitting) on a panel of 40 economies (2000-2023, $N=840$ after excluding missing CO$_2$ outcomes), we uncover striking \textbf{non-linear structural heterogeneity} in the climate impact of digital transformation. We propose a \textbf{"Two-Dimensional Digitalization"} framework that disentangles \textit{Domestic Digital Capacity (DCI)} from \textit{External Digital Specialization (EDS)}. Our forest-based \textbf{Model Ladder} analysis reveals that linear models understate the decarbonization potential of domestic capacity by roughly 1.7x. We shift focus from pointwise significance to \textbf{Group Average Treatment Effects (GATEs)} validated via country-cluster bootstrap. Results suggest a ``Sweet Spot'' in middle-income economies where DCI drives substantial emission reductions, while high-income contexts show weaker (but still negative) effects. These findings provide a high-resolution policy map that linear models miss.
    \end{abstract}
    
    \vspace{1cm}
    
    \noindent\textbf{Keywords:} Causal Forest, Double Machine Learning, Heterogeneous Treatment Effects, Socio-economic Capacity, Economic Development, Institutional Quality
    
    \vspace{0.5cm}
    
    \noindent\textbf{JEL Codes:} C14, C23, O33, Q56
    
\end{titlepage}

\newpage
\onehalfspacing

%==============================================================================
\section{Introduction}
%==============================================================================

The potential of the digital economy to drive environmental sustainability is a subject of intense debate. While digitalization offers pathways to dematerialization and efficiency, it also entails a growing energy footprint from data centers, network infrastructure, and electronic devices \citep{Lange2020}. Previous empirical studies have produced mixed results, often constrained by small sample sizes, omitted variable bias, and linear functional form assumptions \citep{Salahuddin2016}.

\textbf{This paper proposes a new structural perspective: ``Two-Dimensional Digitalization.''} We argue that previous contradictions arise from conflating \textit{Domestic Digital Capacity (DCI)}---the infrastructure to use green technologies---with \textit{External Digital Specialization (EDS)}---the trade role in the global value chain. Our key insight is that while DCI drives decarbonization (hitting a ``sweet spot'' in middle-income economies), high EDS in wealthy nations can weaken reductions.

\subsection{Related Literature}

The relationship between ICT and carbon emissions has been examined through multiple theoretical lenses. The \textit{dematerialization hypothesis} posits that digital technologies substitute for physical products and travel, reducing material throughput and emissions \citep{Berkhout2000}. Conversely, the \textit{rebound effect hypothesis} suggests that efficiency gains from ICT are offset by increased consumption \citep{Gossart2015}.

Empirical evidence remains mixed. \citet{Salahuddin2016} find a positive relationship between ICT and emissions in OECD countries. \citet{Danish2017} report a negative effect in emerging economies. Recent meta-analyses highlight that results are highly sensitive to sample selection, variable definitions, and estimation methods \citep{Lange2020}.

A critical methodological gap is the assumption of linearity and homogeneous treatment effects. Traditional panel models estimate an \textit{average} effect or a \textit{linear interaction}. \textbf{This paper addresses this gap by employing Causal Forest DML \citep{Athey2019}.}

\subsection{The Necessity of Causal Forests}

Why use a machine learning ``cannon'' when a linear interaction model might suffice? We demonstrate that linear models, while capable of detecting the \textit{direction} of heterogeneity, fail to:
\begin{enumerate}
    \item \textbf{Identify Thresholds:} Detect non-linear tipping points where policy effectiveness reverses.
    \item \textbf{Map Off-Diagonal Exceptions:} Capture countries that defy the general trend (e.g., high-income nations with weaker reductions).
    \item \textbf{Provide Rigorous Policy Maps:} Generate decision-relevant strata (GATEs) robust to high-dimensional confounding.
\end{enumerate}

We establish this necessity through a \textbf{``Model Ladder''} comparison, showing where and why linear approximations break down.

\subsection{Contributions}

This study advances the literature in three ways:

\begin{enumerate}
    \item \textbf{Discovery of the ``2D Digital Decarbonization Divide'':} We disentangle the effects of Domestic Capacity (DCI) and External Specialization (EDS). We find that DCI reduces emissions significantly ($-0.96$ tons/capita per SD) but non-linearly, with diminishing returns or reversals in specific high-EDS contexts.
    
    \item \textbf{Causal Forest methodology:} We implement CausalForestDML with 2,000 trees, XGBoost first-stage models, and proper inference intervals---a significant methodological upgrade from linear DML.
    
    \item \textbf{Policy-relevant heterogeneity:} Our results identify which countries benefit from digital decarbonization and which show weaker reductions, enabling targeted policy recommendations.
\end{enumerate}

\subsection{Key Findings}

Our Causal Forest analysis reveals:

\begin{table}[H]
\centering
\caption{Key Findings Summary}
\begin{tabular}{ll}
\toprule
Finding & Value \\
\midrule
Causal Forest ATE (DCI) & $-1.73$ metric tons/capita (per SD) \\
Descriptive diagnostic: $\tau(x)$ vs GDP & $r = -0.16$ \\
Descriptive diagnostic: $\tau(x)$ vs Institution & $r = -0.09$ \\
\bottomrule
\end{tabular}
\end{table}

The remainder of this paper is organized as follows. Section 2 describes the data. Section 3 presents the methodology. Section 4 reports results. Section 5 discusses implications. Section 6 concludes.

%==============================================================================
\section{Data and Sample Construction}
%==============================================================================

\subsection{Data Source}

We retrieve data from two World Bank databases: the World Development Indicators (WDI) and the Worldwide Governance Indicators (WGI). The WDI provides 60 economic, social, and environmental variables, while the WGI offers six dimensions of institutional quality.

\subsection{Sample Selection}

Our sample comprises a focused group of 40 major economies (20 OECD, 20 non-OECD) observed from 2000 to 2023. We employ fold-safe MICE (Multiple Imputation by Chained Equations) for controls and moderators only, fitted within training folds to avoid leakage; outcome and treatment are never imputed. This process yields:
\begin{itemize}
    \item A panel of 40 countries over 24 years.
    \item Analysis sample $N = 840$ after excluding missing CO$_2$ outcomes.
\end{itemize}

\noindent We intentionally focus on 40 major economies to ensure \textbf{high measurement reliability} for server-based indicators used in DCI construction. Expanding coverage to smaller economies would substantially increase measurement noise in secure-server series, potentially degrading the PCA-based capacity measure. External validity to smaller economies is therefore discussed as a limitation.

\subsection{Variables}

\begin{table}[H]
\centering
\caption{Variable Definitions}
\label{tab:variables}
\begin{tabular}{lll}
\toprule
Variable & Definition & Source \\
\midrule
\multicolumn{3}{l}{\textbf{Core Variables}} \\
CO$_2$ Emissions & Per capita (metric tons) & WDI \\
\textbf{Domestic Digital Capacity (DCI)} & PCA Index (Internet, Fixed Broadband, Secure Servers) & WDI (fold-safe MICE) \\
\textbf{External Digital Specialization (EDS)} & ICT service exports (\% of service exports) & WDI \\
\multicolumn{3}{l}{\textbf{Institutional Quality (WGI)}} \\
Control of Corruption & Perceptions of public power for private gain & WGI \\
Rule of Law & Confidence in societal rules & WGI \\
Government Effectiveness & Quality of public services & WGI \\
Regulatory Quality & Private sector development capacity & WGI \\
\multicolumn{3}{l}{\textbf{Control Variables (57 total)}} \\
GDP per capita & Constant 2015 US\$ & WDI \\
Energy Use & Kg oil equivalent per capita & WDI \\
Renewable Energy & \% of total energy consumption & WDI \\
Urban Population & \% of total population & WDI \\
\bottomrule
\end{tabular}
\end{table}

\subsection{Descriptive Statistics}

\begin{table}[H]
\centering
\caption{Descriptive Statistics ($N = 840$)}
\label{tab:desc}
\begin{tabular}{lrrrr}
\toprule
Variable & Mean & Std. Dev. & Min & Max \\
\midrule
CO$_2$ Emissions (metric tons/cap) & 4.61 & 4.45 & 0.04 & 21.87 \\
EDS (ICT Service Exports, \%) & 8.67 & 9.08 & 0.42 & 52.09 \\
Control of Corruption & 0.58 & 1.15 & $-1.60$ & 2.46 \\
GDP per capita (US\$) & 25,580 & 22,174 & 621 & 87,124 \\
Renewable Energy (\%) & 21.84 & 18.54 & 0.00 & 88.10 \\
\bottomrule
\end{tabular}
\end{table}

\noindent\textit{Note: DCI is a composite index (mean=0, sd=1) constructed via PCA from Internet Users, Fixed Broadband Subscriptions, and Secure Servers (Source: WDI with fold-safe MICE, author’s computation). EDS represents the country's export specialization in ICT services.}

%==============================================================================
\section{Methodology}
%==============================================================================

\subsection{From Linear DML to Causal Forest}

Traditional Double Machine Learning \citep{Chernozhukov2018} estimates an \textit{average} treatment effect $\theta$. However, this approach masks heterogeneity. \textbf{Causal Forest DML} \citep{Athey2019} extends this framework to estimate observation-specific effects:

\begin{equation}
\tau(x) = \mathbb{E}[Y(1) - Y(0) | X = x]
\end{equation}

where $\tau(x)$ is the Conditional Average Treatment Effect (CATE) for observation with characteristics $x$.

\subsection{Causal Forest Implementation}

We implement CausalForestDML \citep{Athey2019} with a strict separation of variables to avoid overfitting:

\begin{enumerate}
    \item \textbf{Moderators (X):} A parsimonious set of six theoretical drivers of heterogeneity: \textbf{GDP per capita, EDS, Control of Corruption, Energy Use, Renewable Energy share, Urban Population}. \textit{(Internet Users, Fixed Broadband, and Secure Servers are used exclusively as DCI components and are therefore excluded from X and W to avoid “bad control” concerns.)}
    \item \textbf{Controls (W):} A high-dimensional vector (50+ variables) to capture confounding.
\end{enumerate}

Configuration for Rigor:
\begin{itemize}
    \item n\_estimators: 2,000 trees
    \item Splitting: Honest (to separate training and estimation samples)
    \item Cross-Fitting: GroupKFold (by Country) to prevent temporal leakage
    \item Inference: Cluster Bootstrap (resampling countries)
\end{itemize}

\subsection{Rigorous Inference Strategy}

Instead of relying on unadjusted pointwise confidence intervals, we focus on \textbf{Group Average Treatment Effects (GATEs)}. We stratify the sample by moderator quartiles (e.g., GDP) and compute the ATE within each stratum. Uncertainty is quantified using a \textbf{country-cluster bootstrap ($B=1000$)}, resampling countries with replacement to construct 95\% confidence intervals that account for within-country dependence.

\subsection{Model Ladder}

To justify the model choice, we compare four specifications:
\begin{enumerate}
    \item \textbf{L0 (Baseline):} Two-Way Fixed Effects.
    \item \textbf{L1 (Linear DML):} Global ATE with high-dimensional controls.
    \item \textbf{L2 (Interactive DML):} Linear DML allowing linear moderation by GDP.
    \item \textbf{L3 (Causal Forest):} Full non-linear heterogeneity.
\end{enumerate}

%==============================================================================
\section{Empirical Results}
%==============================================================================

\subsection{Heterogeneity Verification (Phase 1)}

Before running the full Causal Forest, we verify heterogeneity exists using an interaction term model:

\begin{equation}
Y = \beta_1 T + \beta_2 (T \times M) + g(W) + \epsilon
\end{equation}

where $M$ is log(GDP per capita). We also test institutional quality as a moderator.

\begin{table}[H]
\centering
\caption{Interaction Term Results}
\begin{tabular}{lllccc}
\toprule
Moderator & Coefficient & Estimate & SE & $p$-value \\
\midrule
log(GDP) & Main Effect (DCI) & $-3.365$ & 0.201 & $<0.001$ \\
log(GDP) & Interaction (DCI $\times$ log(GDP)) & $-0.126$ & 0.128 & 0.326 \\
Institution & Main Effect (DCI) & $-1.376$ & 0.258 & $<0.001$ \\
Institution & Interaction (DCI $\times$ Institution) & \textbf{0.765} & 0.190 & \textbf{$<0.001$} \\
\bottomrule
\end{tabular}
\end{table}

The GDP interaction term is not statistically significant ($p = 0.326$). The institutional quality interaction is \textbf{statistically significant} ($p < 0.001$).

\subsection{The Model Ladder: Why Non-Linearity Matters}

We estimate treatment effects across four increasingly flexible specifications to demonstrate the necessity of the Causal Forest approach.

\begin{table}[H]
\centering
\caption{Model Ladder Comparison (DCI Effect, B=1000)}
\label{tab:ladder}
\begin{tabular}{llcccc}
\toprule
Model & ATE Estimate (per SD) & SE & 95\% CI & Heterogeneity Caught? \\
\midrule
\textbf{L0} (TWFE) & $-2.810$ & 0.463 & $[-3.727, -1.844]$ & None \\
\textbf{L1} (Linear DML) & $-0.990$ & 0.436 & $[-2.070, -0.367]$ & None \\
\textbf{L2} (Interactive) & $-1.216$ & 0.393 & $[-2.031, -0.460]$ & Linear Only \\
\textbf{L3} (\textbf{Causal Forest}) & \textbf{$-1.730$} & \textbf{0.588} & \textbf{$[-2.882, -0.578]$} & \textbf{Complex} \\
\bottomrule
\end{tabular}
\end{table}

\noindent\textbf{Key Insight:} Linear models systematically understate the decarbonization potential of domestic capacity (finding $\sim -0.99$ tons/SD). The Causal Forest reveals a \textbf{larger effect} ($-1.73$ tons/SD) by correctly identifying the high-impact ``sweet spots'' that linear averages smooth over.

\subsection{Group Average Treatment Effects (GATEs)}

Instead of relying on specific point estimates, we report GATEs stratified by GDP quartiles, with 95\% confidence intervals derived from \textbf{cluster bootstrapping}.

\begin{table}[H]
\centering
\caption{GATE Results (DCI Effect, B=1000)}
\label{tab:gate}
\begin{tabular}{lcll}
\toprule
GDP Group & Estimate (per SD) & 95\% CI & Interpretation \\
\midrule
\textbf{Low Income} & \textbf{$-1.19$} & $[-1.47, -0.99]$ & Effective \\
\textbf{Lower-Mid} & \textbf{$-2.17$} & $[-2.66, -1.76]$ & \textbf{Sweet Spot} \\
\textbf{Upper-Mid} & \textbf{$-2.29$} & $[-2.65, -1.85]$ & \textbf{Sweet Spot} \\
\textbf{High Income} & \textbf{$-1.26$} & $[-1.67, -0.81]$ & Weaker Effects \\
\bottomrule
\end{tabular}
\end{table}

The transition reveals a \textbf{``Sweet Spot''} in middle-income economies where domestic digital capacity delivers the largest carbon reductions. In high-income economies, the effect weakens but remains negative.

\subsubsection{Robustness: Placebo \& LOCO}
\begin{itemize}
    \item \textbf{Placebo Test:} Permuting treatment yields a CATE SD of \textbf{0.041} (vs Real SD \textbf{0.952}), implying a Signal-to-Noise ratio of $\sim 23\times$.
    \item \textbf{LOCO Stability:} Leave-One-Country-Out analysis confirms robustness. The Global ATE remains significant in every fold (Range: $-2.33$ to $-0.67$), proving results are not driven by any single outlier.
\end{itemize}

\subsubsection{Policy Exceptions (Weakest Reductions)}
Correctly measuring Domestic Capacity (DCI) shows uniformly negative country-average effects, but with meaningful variation in magnitude. The weakest reductions appear in a small set of high-income countries.

\begin{table}[H]
\centering
\caption{Policy Exceptions (Weakest Reductions)}
\label{tab:exceptions}
\begin{tabular}{lcccl}
\toprule
Country & Forest CATE (DCI) & 95\% CI & Verdict \\
\midrule
\textbf{FIN} & $-0.19$ & $[-0.39, -0.07]$ & Weakest Reduction \\
\textbf{SWE} & $-0.46$ & $[-0.60, -0.30]$ & Weak Reduction \\
\textbf{CHE} & $-0.50$ & $[-0.57, -0.44]$ & Weak Reduction \\
\textbf{CAN} & $-0.52$ & $[-0.62, -0.44]$ & Weak Reduction \\
\textbf{VNM} & $-0.90$ & $[-1.01, -0.82]$ & Moderate Reduction \\
\bottomrule
\end{tabular}
\end{table}

\subsection{Sources of Heterogeneity}

\begin{table}[H]
\centering
\caption{Correlation between CATE and Moderators}
\begin{tabular}{lcl}
\toprule
Moderator & Correlation (r) & Interpretation \\
\midrule
GDP per capita (log) & $-0.33$ & Higher GDP $\to$ stronger reduction \\
Energy use per capita & $-0.64$ & \textbf{Strongest predictor} \\
Control of Corruption & $-0.09$ & Weak positive alignment \\
Renewable energy \% & $+0.56$ & Higher renewables $\to$ weaker reduction \\
\bottomrule
\end{tabular}
\end{table}
\textit{Note: Correlations are computed between estimated CATEs and moderators and are descriptive; they do not account for within-country dependence.}

\subsection{Visualizing the Divide}

\subsubsection{Figure 1: Why Linear Models Fail}
\begin{figure}[H]
\centering
\includegraphics[width=0.85\linewidth]{results/figures/linear_vs_forest.png}
\caption{Panel A compares the linear interaction model (dashed line) with the flexible Causal Forest estimation (solid line). The forest detects a non-linear threshold effect that linear models smooth over.}
\label{fig:linear_vs_forest}
\end{figure}

\subsubsection{Figure 2: The Off-Diagonal Analysis}
\begin{figure}[H]
\centering
\includegraphics[width=0.85\linewidth]{results/figures/off_diagonal_cis.png}
\caption{Panel B identifies ``Policy Exceptions''---countries where the Forest prediction deviates from the Linear prediction. The weakest reductions are concentrated in a small set of high-income countries.}
\label{fig:off_diagonal}
\end{figure}

\subsubsection{Figure 3: Group Average Treatment Effects (GATEs)}
\begin{figure}[H]
\centering
\includegraphics[width=0.85\linewidth]{results/figures/gate_plot.png}
\caption{Group Average Treatment Effects with 95\% Cluster-Bootstrap Confidence Intervals. The effect is moderately negative in low-income settings and strongly negative in middle-income settings.}
\label{fig:gate}
\end{figure}

%==============================================================================
\section{Discussion}
%==============================================================================

\subsection{The Digital Decarbonization Divide}

Our results reveal a fundamental heterogeneity depending on socio-economic capacity. The ``Digital Decarbonization Divide'' manifests along three dimensions:

\begin{enumerate}
    \item \textbf{Development Divide (with Exceptions):} Wealthier nations \textit{generally} benefit more from domestic digital capacity (GATEs confirm strong reductions in the top quartiles), yet a subset (e.g., FIN, SWE, CHE, CAN) exhibits notably weaker reductions.
    \item \textbf{EDS Alignment:} Higher EDS is associated with weaker reductions (positive correlation), suggesting export structure can dampen domestic efficiency gains without reversing them.
    \item \textbf{Energy Structure Divide:} Counterintuitively, countries with \textit{lower} renewable energy shares see stronger DCI-driven reductions.
\end{enumerate}

\subsection{Mechanism Interpretation}

We propose two non-mutually exclusive mechanisms:

\textbf{Enabling Conditions Hypothesis}: Strong institutions enable effective environmental regulation, ensuring that efficiency gains from ICT translate to emission reductions rather than diminished benefits.

\textbf{Structural Transformation Hypothesis}: ICT development in wealthy economies represents a shift toward service-based, knowledge-intensive production that is inherently less carbon-intensive.

\subsection{Policy Implications}

\textbf{For Developed Economies:}
\begin{quote}
The aggregate trend suggests \textbf{domestic digital capacity (DCI)} can be a decarbonization lever. However, \textbf{high-EDS structural exceptions} indicate that efficiency gains may be weaker in specific service-export-intensive contexts. Policy should therefore complement digital investment with measures targeting \textbf{absolute decoupling}.
\end{quote}

\textbf{For Developing Economies:} 
\begin{quote}
\textbf{Policy Consideration:} Evidence suggests that digital transformation alone may not drive decarbonization in low-capacity settings. Complementary efforts in capacity building are essential.
\end{quote}

\textbf{For International Organizations:} Target digital development assistance as part of broader \textbf{capacity-building} packages.

\subsection{Limitations}

\begin{itemize}
    \item \textbf{Measurement:} While \textbf{DCI} (PCA of internet use, broadband access, and secure servers) captures infrastructure-based domestic digital capacity, it may not fully capture the \textbf{quality of digital utilization} (e.g., AI adoption intensity, data-center efficiency, sectoral digital deepening). \textbf{EDS} captures external specialization and should not be interpreted as a proxy for domestic adoption.
    \item \textbf{Causal interpretation}: Despite the DML framework, unobserved confounders may remain.
    \item \textbf{External validity}: Results may not generalize to small economies not in our sample. Accordingly, inferential statements are restricted to GATE-level contrasts with country-cluster bootstrap intervals.
\end{itemize}

%==============================================================================
\section{Conclusion}
%==============================================================================

This paper introduces the concept of the ``Digital Decarbonization Divide'' and provides rigorous empirical evidence for its existence. Using Causal Forest DML on a panel of 40 economies ($N=840$ after excluding missing CO$_2$ outcomes), we find that:

\begin{enumerate}
    \item \textbf{Domestic digital capacity (DCI)} exhibits fundamentally non-linear effects on CO$_2$ emissions.
    \item \textbf{GATEs} reveal a clear progression from near-zero effects in low-capacity economies to strong reductions in high-capacity economies.
    \item \textbf{Structural exceptions exist:} ``off-diagonal'' cases indicate that \textbf{high external digital specialization (EDS)} can dampen domestic efficiency gains.
    \item The \textbf{Model Ladder} demonstrates that flexible estimation is required to capture policy-relevant thresholds and exceptions missed by linear models.
\end{enumerate}

Our findings challenge the assumption that digital transformation is universally beneficial for climate goals. Instead, we identify \textbf{conditional prerequisites}---socio-economic capacity---that appear to moderate whether ICT delivers a ``green dividend.''

%==============================================================================
\section*{Declarations}
%==============================================================================

\textbf{Funding}: This research did not receive any specific grant from funding agencies in the public, commercial, or not-for-profit sectors.

\textbf{Conflicts of Interest}: The author declares no conflicts of interest.

%==============================================================================
\section*{Data and Code Availability}
%==============================================================================

Replication code and a cleaned dataset construction script will be made available in a public repository upon acceptance. All raw inputs are from WDI/WGI.

%==============================================================================
\bibliographystyle{apalike}
\begin{thebibliography}{99}

\bibitem[Athey and Wager(2019)]{Athey2019}
Athey, S. and Wager, S. (2019).
\newblock Estimating treatment effects with causal forests: An application.
\newblock \textit{Observational Studies}, 5(2), 37--51.

\bibitem[Berkhout and Hertin(2000)]{Berkhout2000}
Berkhout, F. and Hertin, J. (2000).
\newblock De-materialising the economy or rematerialisation? The case of information and communication technologies.
\newblock \textit{The Environmental Impact of Prosperous Societies}, 4, 12--26.

\bibitem[Chernozhukov et al.(2018)]{Chernozhukov2018}
Chernozhukov, V., Chetverikov, D., Demirer, M., Duflo, E., Hansen, C., Newey, W., and Robins, J. (2018).
\newblock Double/debiased machine learning for treatment and structural parameters.
\newblock \textit{The Econometrics Journal}, 21(1), C1--C68.

\bibitem[Danish et al.(2017)]{Danish2017}
Danish, Zhang, B., Wang, B., and Wang, Z. (2017).
\newblock Role of renewable energy and non-renewable energy consumption on EKC: Evidence from Pakistan.
\newblock \textit{Journal of Cleaner Production}, 156, 855--864.

\bibitem[Gossart(2015)]{Gossart2015}
Gossart, C. (2015).
\newblock Rebound effects and ICT: A review of the literature.
\newblock In \textit{ICT Innovations for Sustainability} (pp. 435--448). Springer.

\bibitem[Lange et al.(2020)]{Lange2020}
Lange, S., Pohl, J., and Santarius, T. (2020).
\newblock Digitalization and energy consumption. Does ICT reduce energy demand?
\newblock \textit{Ecological Economics}, 176, 106760.

\bibitem[Salahuddin and Alam(2016)]{Salahuddin2016}
Salahuddin, M. and Alam, K. (2016).
\newblock ICT, electricity consumption and economic growth in OECD countries.
\newblock \textit{International Journal of Electrical Power \& Energy Systems}, 76, 185--193.

\bibitem[World Bank(2026)]{WorldBank2026}
World Bank. (2026).
\newblock \textit{World Development Indicators}.
\newblock Washington, D.C.: The World Bank.

\end{thebibliography}

\end{document}
