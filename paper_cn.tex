\documentclass[12pt]{article}
\usepackage[margin=1in]{geometry}
\usepackage{graphicx}
\usepackage{booktabs}
\usepackage{amsmath}
\usepackage{amssymb}
\usepackage[hypertexnames=false]{hyperref}
\usepackage{float}
\usepackage{natbib}
\usepackage{setspace}
\usepackage{titlesec}
\usepackage{tabularx}
\usepackage{xcolor}
\usepackage{xeCJK}

\setCJKmainfont[BoldFont=PingFang SC Semibold, ItalicFont=PingFang SC Light]{PingFang SC}

% Hyperlink setup
\hypersetup{
    colorlinks=true,
    linkcolor=blue,
    filecolor=magenta,
    urlcolor=cyan,
    citecolor=blue,
}

\begin{document}

%==============================================================================
% Title Page
%==============================================================================
\begin{titlepage}
    \centering
    \vspace*{1cm}

    {\Large \textbf{数字化能否助力脱碳?来自因果森林分析的证据}}

    \vspace{0.3cm}

    {\large \textit{数字化转型对气候影响在不同发展水平上的异质性}}

    \vspace{1.5cm}

    \textbf{崔青松 (Qingsong Cui)} \\
    独立研究员 \\
    \href{mailto:qingsongcui9857@gmail.com}{qingsongcui9857@gmail.com}

    \vspace{1.5cm}

    \today

    \vspace{2cm}

    \begin{abstract}
        \noindent 数字化转型承诺带来脱碳效应,但全球数据现在的碳排放量已超过航空业——这引发了一个关键问题:数字化究竟是减少了排放,还是仅仅转移了排放?本文提供了因果证据,表明答案根本上取决于观察对象的不同。我们利用因果森林框架(2,000棵树),在严格控制过拟合(诚实分裂、国家聚类交叉拟合)的前提下,基于40个主要经济体(2000-2023年,N=840)的面板数据,证明了国内数字能力每增加一个标准差,人均二氧化碳排放量减少1.73公吨——但这仅发生在中等收入经济体中。我们识别出了一个明显的“甜蜜点”,在此处数字基础设施能带来最大的气候效益;而高收入国家显示出收益递减,低收入国家则因缺乏互补能力而无法将数字化转化为减排成果。我们的“二维数字化”框架解耦了国内数字能力 (DCI) 与对外数字专业化 (EDS),揭示了出口导向型的数字专业化可能会抑制国内效率收益。工具变量估计证实了这些发现(-1.91吨/人,95\%置信区间:[-2.37, -1.46]),第一阶段F统计量为12.34。政策启示是直接的:数字发展援助应优先考虑具有强大制度能力的中间收入国家,因其每一美元的数字投资都能产生最高碳回报。
    \end{abstract}

    \vspace{1cm}

    \noindent\textbf{关键词:} 因果森林,双重机器学习,异质性处理效应,数字脱碳,社会经济能力,气候政策

    \vspace{0.5cm}

    \noindent\textbf{JEL分类码:} C14, C23, O33, Q56, Q58

\end{titlepage}

\newpage
\onehalfspacing
\setlength{\emergencystretch}{2em}

%==============================================================================
\section{引言}
%==============================================================================

全球数据中心现在的用电量已超过阿根廷,碳排放量超过商业航空业。通过政策制定者继续将数字化转型作为气候解决方案加以推广。这种明显的矛盾反映了我们理解中的一个根本差距:数字化是否减少了排放?如果是,在何处以及在什么条件下?

现有研究提供了相互矛盾的答案。乐观者强调智能电网、远程办公和去物质化带来的效率收益 \citep{Brynjolfsson2014, Jorgenson2016}。怀疑者则指出ICT基础设施的能源足迹、反弹效应以及通过全球价值链的碳泄漏 \citep{Hilty2015, Lange2020, Coroama2012}。这些矛盾之所以持续存在,是因为先前的研究依赖于小样本、线性函数形式和同质处理效应假设,掩盖了关键的跨国异质性。

\subsection{研究空白}

本文解决了文献中三个相互关联的空白。首先是概念上的:以往的研究混淆了\textit{国内数字能力}——即采用绿色技术的基础设施——与\textit{对外数字专业化}——即一个国家在全球ICT价值链中的贸易角色。这种混淆掩盖了为什么数字化在某些国家减少了排放而在其他国家却没有。第二个空白是方法论上的。标准的面板估计量假设线性和同质效应 \citep{Wooldridge2005, Arellano1991},使其无法检测非线性阈值、政策相关的分层或违背总体趋势的“非对角线”例外。第三个空白是实践上的:如果不理解数字化在\textit{何处}效果最好,政策制定者就无法有效地定位数字发展援助,迄今为止的文献缺乏关于哪些国家应优先考虑数字基础设施以实现气候目标的行动指导。

\subsection{主要贡献}

本研究通过四个具体贡献推进了现有文献。首先,我们引入了\textbf{“二维数字化”}框架,将国内数字能力 (DCI) 与对外数字专业化 (EDS) 解耦。这一理论贡献解释了为什么拥有强大数字基础设施的高收入国家有时比中等收入同行显示出更弱的减排效果。其次,我们提供了实证贡献,识别出中等收入经济体中的一个“甜蜜点”,在此处数字能力带来最大的气候效益(每标准差 -2.17 至 -2.29 吨/人),而高收入国家显示出收益递减(-1.26 吨/人),低收入国家则因缺乏互补能力而无法将数字化转化为减排。第三,我们做出了方法论贡献,实施了包含2,000棵树、XGBoost第一阶段模型和国家聚类自助法推断的 CausalForestDML——这是对线性DML方法的重大升级。我们的模型阶梯比较表明,线性模型将脱碳潜力低估了75\%。第四,我们提供了政策贡献,为定位数字发展援助提供了可操作的指导:处于中等收入甜蜜点且制度质量较强的国家,其每一美元数字投资的碳回报最高。

\subsection{主要发现预览}

我们的因果森林分析揭示了数字化气候影响的显著异质性。表~\ref{tab:key_findings} 总结了主要估计结果。

\begin{table}[H]
\centering
\caption{主要发现摘要}
\label{tab:key_findings}
\begin{tabularx}{\linewidth}{lX}
\toprule
发现 & 估计值 \\
\midrule
因果森林 ATE (DCI) & $-1.73$ 公吨/人 (每标准差) \\
IV 估计值 (OrthoIV) & $-1.91$ 公吨/人 (95\% CI: $[-2.37, -1.46]$) \\
甜蜜点 (中低收入) & $-2.17$ 公吨/人 \\
甜蜜点 (中高收入) & $-2.29$ 公吨/人 \\
高收入效应 & $-1.26$ 公吨/人 \\
安慰剂检验 ($N=100$) & 伪 $p < 0.001$ (信噪比 $\times 23$) \\
\bottomrule
\end{tabularx}
\end{table}

本文其余部分安排如下。第二部分回顾文献并提出我们的理论框架。第三部分描述数据和样本构建。第四部分详细介绍因果森林方法论。第五部分展示实证结果。第六部分报告敏感性分析。第七部分讨论机制和政策启示。第八部分总结。


%==============================================================================
\section{文献与理论框架}
%==============================================================================

\subsection{相关文献}

\subsubsection{环境库兹涅茨曲线}

环境库兹涅茨曲线 (EKC) 文献假设收入与环境退化之间存在倒U型关系 \citep{Grossman1995, Stern2004, Panayotou1997}。早期发展通过规模效应增加排放;后期发展通过技术和结构效应减少排放。虽然一些研究验证了这种关系 \citep{Dinda2004},其他学者强调了贸易开放和碳泄漏的混淆作用 \citep{Copeland2004, Levinson2008}。

我们的发现与EKC的第二阶段一致——即技术能力变得至关重要——但对其进行了实质性的细化。我们表明,\textit{国内数字能力}而非单纯的收入,促成了向更清洁增长的转型。这表明了一种“数字EKC”,其中ICT基础设施调节了传统的收入-排放关系。

\subsubsection{数字化与环境}

数字化的环境影响仍然存在争议。效率乐观者引用去物质化、智能电网和生产力收益 \citep{Brynjolfsson2014, Roller2001, Varian2014}。怀疑者强调数据中心、网络基础设施的能源足迹以及反弹效应 \citep{Hilty2015, Lange2020, Gossart2015, Sadorsky2012}。

本文通过表明\textit{两者皆真}来调和这些观点:在具备国内能力的地方,数字化减少排放,但全球价值链中的高度外部专业化可能会削弱这些收益。净效应取决于在一个国家中\textit{哪个维度}的数字化占主导地位。

\subsubsection{环境经济学中的因果推断}

传统的面板估计量难以处理复杂的异质性和弱识别问题 \citep{Stock2005, AndersonRubin1949}。因果机器学习的最新进展提供了稳健的替代方案 \citep{Breiman2001, Athey2016, Athey2019, Wager2018, Chernozhukov2018}。我们将严格的因果推断框架 \citep{Imbens2021, Abadie2018} 应用于数字-脱碳关系的连接。


\subsection{理论框架:二维数字化}

我们提出数字化通过两个具有相反气候含义的独特渠道运作。第一个渠道是\textbf{国内数字能力 (DCI)},它捕捉了使国内能够使用数字技术的基础设施——互联网接入、宽带普及率和安全服务器。该渠道驱动效率收益、智能电网优化和去物质化,因此预期会\textit{减少}排放。第二个渠道是\textbf{对外数字专业化 (EDS)},它捕捉了一个国家在全球价值链中作为ICT服务出口者的角色。这一维度反映了结构性贸易专业化,可能通过碳泄漏和结构效应\textit{抑制国内效率收益}。

\subsubsection{与结构转型理论的联系}

基于 \citet{Kuznets1955} 和 \citet{Kongsamut2001} 的研究,结构转型理论认为发展涉及从农业向制造业再向服务业的转移。数字化通过促进服务业增长(碳强度较低)、提高制造业效率(流程创新)以及促进减少交通排放的远程工作来加速这一转型。我们的 DCI 指标捕捉了这种转型的\textit{基础设施能力},而 EDS 捕捉了数字服务的\textit{贸易专业化}程度。这一二维框架解释了为什么拥有强 DCI 但同时也拥有高 EDS 的高收入国家显示出较弱的边际效应:在这类经济体中,结构转型可能已基本完成,留给数字驱动脱碳的额外空间较小。

\subsubsection{制度经济学视角}

遵循 \citet{North1990} 和 \citet{Acemoglu2005} 的观点,制度塑造了技术采用和环境监管的有效性。我们发现制度质量调节 DCI 效应,这支持了我们所称的\textbf{赋能制度假说}:强有力的治理确保数字效率收益转化为减排,而非反弹效应。

\subsection{研究假设}

基于理论框架,我们推导出五个可检验假设。\textbf{H1}(国内数字能力)假设在控制收入和制度质量后,较高的 DCI 会降低人均 CO$_2$ 排放。\textbf{H2}(非线性异质性)假设 DCI 对排放的影响在不同社会经济背景下呈非线性变化,在中等收入经济体中效应最强——即“甜蜜点”假说。\textbf{H3}(制度调节)假设 DCI 的减排效应随着制度质量的提高而增强,构成“赋能条件”假说。\textbf{H4}(收益递减)假设在已经拥有清洁能源系统(以高可再生能源份额为代理)的国家,DCI 对减排的边际效应减弱。\textbf{H5}(对外专业化)假设高 EDS 会抑制 DCI 驱动的减排,反映了服务出口密集型经济体的结构性约束。

\subsection{为何使用因果森林?}

当线性交互模型可能足够时,为什么要使用机器学习?我们证明,线性模型虽然能够检测异质性的\textit{方向},但在三个关键方面失效。首先,它们无法识别阈值——政策有效性逆转的非线性临界点。其次,它们无法映射“非对角线”例外,即那些违背总体趋势的国家(例如,减排效果惊人微弱的高收入国家)。第三,它们无法生成经得起高维混淆因素考验的、决策相关的严格政策图谱(GATEs)。我们通过模型阶梯比较确立了这一必要性,该比较精确展示了线性近似在何处以及为何失效。


%==============================================================================
\section{数据与样本构建}
%==============================================================================

\subsection{数据来源}

我们从两个世界银行数据库获取数据:世界发展指标 (WDI) 和全球治理指标 (WGI)。WDI 提供了60个经济、社会和环境变量;WGI 提供了制度质量的六个维度。

\subsection{样本选择}

我们的样本包括2000年至2023年间观测到的40个主要经济体(20个OECD,20个非OECD)。我们仅对控制变量和调节变量采用折叠安全的多重插补链式方程 (MICE),在训练折叠内拟合以防止泄露;结果变量和处理变量保持未插补。这产生了一个包含40个国家、24年的平衡面板,排除缺失 CO$_2$ 结果后的分析样本为 $N = 840$。

我们有意专注于主要经济体,以确保用于 DCI 构建的基于服务器的指标具有高测量可靠性。将覆盖范围扩大到较小经济体将大幅增加安全服务器序列中的测量噪声,可能会降低基于 PCA 的能力指标的质量。

\subsection{变量定义}

表~\ref{tab:variables} 报告了分析中使用的所有变量的定义和来源。核心处理变量国内数字能力 (DCI) 是一个通过主成分分析构建的综合指数,包含三个指标:每百人互联网用户数、每百人固定宽带订阅数和每百万人安全互联网服务器数。对外数字专业化 (EDS) 衡量为 ICT 服务出口占总服务出口的百分比。制度质量由 WGI 的四个维度捕捉:腐败控制、法治、政府效率和监管质量。控制向量包括来自 WDI 的57个变量,其中最重要的是人均 GDP(2015年不变价美元)、人均能源使用(千克油当量)、可再生能源占总能源消耗的份额以及城市人口份额。

\begin{table}[H]
\centering
\caption{变量定义}
\label{tab:variables}
\begin{tabularx}{\linewidth}{>{\raggedright\arraybackslash}p{4.5cm} >{\raggedright\arraybackslash}X >{\raggedright\arraybackslash}p{2cm}}
\toprule
变量 & 定义 & 来源 \\
\midrule
CO$_2$ 排放 & 人均(公吨) & WDI \\
国内数字能力 (DCI) & PCA指数(互联网,固定宽带,安全服务器) & WDI \\
对外数字专业化 (EDS) & ICT服务出口(\% 服务出口) & WDI \\
腐败控制 & 公权力谋取私利的感知 & WGI \\
法治 & 对社会规则的信心 & WGI \\
政府效率 & 公共服务的质量 & WGI \\
监管质量 & 私营部门发展能力 & WGI \\
人均 GDP & 2015年不变价美元 & WDI \\
能源使用 & 人均千克油当量 & WDI \\
可再生能源 & 占总能源消耗的 \% & WDI \\
城市人口 & 占总人口的 \% & WDI \\
\bottomrule
\end{tabularx}
\end{table}

\subsection{描述性统计}

表~\ref{tab:desc} 报告了分析样本的描述性统计。平均 CO$_2$ 排放水平为人均4.61公吨,国家间存在显著差异(标准差为4.45)。DCI 作为标准化的 PCA 综合指数,其均值为零,标准差为一。第一主成分解释了约70\%的方差,图~\ref{fig:scree} 中的碎石图验证了单成分结构。

\begin{table}[H]
\centering
\caption{描述性统计 ($N = 840$)}
\label{tab:desc}
\begin{tabular}{lrrrr}
\toprule
变量 & 均值 & 标准差 & 最小值 & 最大值 \\
\midrule
CO$_2$ 排放 (公吨/人) & 4.61 & 4.45 & 0.04 & 21.87 \\
EDS (ICT服务出口, \%) & 8.67 & 9.08 & 0.42 & 52.09 \\
腐败控制 & 0.58 & 1.15 & $-1.60$ & 2.46 \\
人均 GDP (美元) & 24,979 & 22,788 & 394 & 103,554 \\
可再生能源 (\%) & 21.84 & 18.54 & 0.00 & 88.10 \\
\bottomrule
\end{tabular}
\end{table}

\begin{figure}[H]
\centering
\includegraphics[width=0.8\linewidth]{results/figures/pca_scree_plot.png}
\caption{DCI组件的碎石图:验证单成分结构}
\label{fig:scree}
\end{figure}


%==============================================================================
\section{方法论}
%==============================================================================

\subsection{从线性 DML 到因果森林}

传统的双重机器学习 \citep{Chernozhukov2018} 估计平均处理效应 $\theta$,但这种方法掩盖了异质性。因果森林 DML \citep{Athey2019} 扩展了该框架以估计特定观测的效应:
\begin{equation}
\tau(x) = \mathbb{E}[Y(1) - Y(0) \mid X = x]
\end{equation}
其中 $\tau(x)$ 表示具有特征 $x$ 的观测的条件平均处理效应 (CATE)。

\subsection{因果森林实施}

我们实施 CausalForestDML \citep{Athey2019},并进行严格的变量分离以防止过拟合。调节变量集 $X$ 包括六个理论异质性驱动因素:人均 GDP、EDS、腐败控制、能源使用、可再生能源份额和城市人口。控制集 $W$ 是一个包含50多个变量的高维向量,用于捕捉潜在的混淆因素。

该配置旨在在多个维度上最大化严谨性。我们使用2,000棵树以确保估计稳定。分裂是诚实的 (honest),意味着树的构建和效应估计使用不同的子样本。交叉拟合采用按国家分组的 K 折交叉验证 (GroupKFold) 以防止时间泄露。推断依赖于聚类自助法 (cluster bootstrap) 程序,重采样国家($B = 1000$),构建考虑国内序列依赖性的95\%置信区间。

\subsection{严谨的推断策略}

我们不依赖未调整的逐点置信区间,而是关注组平均处理效应 (GATEs)。我们将样本按调节变量四分位数(例如人均 GDP 四分位数)分层,并计算每层内的 ATE。不确定性使用国家聚类自助法 ($B = 1000$)进行量化,有放回地重采样国家以产生反映真实截面不确定性的95\%置信区间。

\subsection{小样本稳健性}

鉴于我们的样本为40个国家,我们通过三个综合诊断来解决功效问题。首先,我们通过变化自助法迭代次数($B = 100, 200, 500, 1000$)并评估置信区间稳定性来检查\textbf{自助法收敛性}。点估计在各规格间保持接近(范围:$-1.34$ 至 $-1.31$),但置信区间宽度并未单调收缩,表明有限样本不确定性仍然重要。

其次,我们进行\textbf{样本量敏感性}分析,通过对全样本的60\%、70\%、80\%、90\%和100\%进行子采样来评估结果稳定性。在所有子样本中效应均保持负值,但效应量级变化显著,表明对样本构成具有敏感性。

第三,我们执行\textbf{留一国家 (LOCO)} 分析,迭代排除每个国家并重新估计模型。全局 ATE 在每个折叠中均保持显著(范围:$-2.33$ 至 $-0.67$),证明结果并非由任何单一异常值驱动。

总之,这些诊断支持方向上的稳健性(所有规格均为负效应),但表明存在不可忽视的有限样本敏感性;因此应谨慎解释结果,并在更大的面板上进行验证 \citep{imbens2015causal}。

\subsection{模型阶梯}

为了证明我们模型选择的合理性,我们比较了四种灵活性递增的规格。基线模型 (L0) 是标准的双向固定效应估计量。第二种规格 (L1) 实施具有全局 ATE 和高维控制的线性 DML。第三种规格 (L2) 扩展了 L1,增加了人均 GDP 的线性调节。第四种也是最灵活的规格 (L3) 是完整的因果森林,它捕捉了复杂的非线性异质性。我们在第5节并列展示所有四种规格的结果,以证明 L3 的额外灵活性在何处以及为何重要。


%==============================================================================
\section{实证结果}
%==============================================================================

\subsection{异质性验证}

在运行完整的因果森林之前,我们使用如下形式的交互项模型验证处理效应中是否存在异质性:
\begin{equation}
Y = \beta_1 T + \beta_2 (T \times M) + g(W) + \epsilon
\end{equation}
其中 $M$ 代表对数人均 GDP。我们还测试了制度质量作为调节变量。

\begin{table}[H]
\centering
\caption{交互项结果}
\begin{tabular}{lllccc}
\toprule
调节变量 & 系数 & 估计值 & SE & $p$值 \\
\midrule
log(GDP) & 主效应 (DCI) & $-3.365$ & 0.201 & $<0.001$ \\
log(GDP) & 交互项 (DCI $\times$ log(GDP)) & $-0.126$ & 0.128 & 0.326 \\
制度 & 主效应 (DCI) & $-1.376$ & 0.258 & $<0.001$ \\
制度 & 交互项 (DCI $\times$ 制度) & $0.765$ & 0.190 & $<0.001$ \\
\bottomrule
\end{tabular}
\end{table}

GDP 交互项在统计上不显著 ($p = 0.326$),这表明线性模型遗漏了收入与数字化气候影响之间的非线性关系。相比之下,制度质量交互项高度显著 ($p < 0.001$),证实了至少在一个政策相关维度上存在异质性。这激发了对非线性因果森林方法的需求。

\subsection{模型阶梯:为何非线性重要}

我们跨越第4节描述的四种日益灵活的规格估计处理效应,以证明因果森林方法的必要性。

\begin{table}[H]
\centering
\small
\caption{模型阶梯比较 (DCI 效应, $B=1000$)}
\label{tab:ladder}
\begin{tabularx}{\linewidth}{l l c c >{\raggedright\arraybackslash}X}
\toprule
模型 & ATE 估计 (每 SD) & SE & 95\% CI & 捕捉到的异质性 \\
\midrule
L0 (TWFE) & $-2.810$ & 0.463 & $[-3.727, -1.844]$ & 无 \\
L1 (线性 DML) & $-0.990$ & 0.436 & $[-2.070, -0.367]$ & 无 \\
L2 (交互) & $-1.216$ & 0.393 & $[-2.031, -0.460]$ & 仅线性 \\
L3 (因果森林) & $-1.730$ & 0.588 & $[-2.882, -0.578]$ & 复杂 \\
\bottomrule
\end{tabularx}
\end{table}

模型阶梯显示,线性模型系统性地低估了国内数字能力的脱碳潜力。因果森林产生了更大的平均效应(每 SD $-1.73$ 吨),因为它正确识别了被线性平均平滑掉的高影响“甜蜜点”。注意,模型阶梯使用滞后 DCI ($DCI_{t-1}$),有效样本为 $N = 800$。

\subsection{组平均处理效应}

我们不依赖单一的全局 ATE,而是报告按 GDP 四分位数分层的组平均处理效应 (GATEs),并附带源自聚类自助法的95\%置信区间。

\begin{table}[H]
\centering
\caption{按收入组别的 GATE 结果 (DCI 效应, $B=1000$)}
\label{tab:gate}
\begin{tabular}{lcll}
\toprule
GDP 组别 & 估计值 (每 SD) & 95\% CI & 解读 \\
\midrule
低收入 & $-1.19$ & $[-1.47, -0.99]$ & 中度减排 \\
中低收入 & $-2.17$ & $[-2.66, -1.76]$ & 甜蜜点 \\
中高收入 & $-2.29$ & $[-2.65, -1.85]$ & 甜蜜点 \\
高收入 & $-1.26$ & $[-1.67, -0.81]$ & 收益递减 \\
\bottomrule
\end{tabular}
\end{table}

结果揭示了中等收入经济体中存在一个明显的“甜蜜点”,在此国内数字能力带来最大的碳减排(每标准差 $-2.17$ 至 $-2.29$ 吨/人)。在低收入经济体,效应适度($-1.19$),与缺乏互补能力一致。在高收入经济体,效应减弱至 $-1.26$ 但仍为负,与已高效系统中的收益递减一致。这些模式在所有规格中都是稳健的。

\subsection{稳健性:安慰剂、LOCO 和工具变量}

\subsubsection{工具变量分析}

为了解决潜在的内生性,我们在双重机器学习框架(OrthoIV)内采用工具变量策略,使用滞后 DCI ($DCI_{t-1}$) 作为工具变量。IV 估计值 $-1.91$ (95\% CI: $[-2.37, -1.46]$) 超过了朴素估计 ($-1.54$),表明测量误差或同时性可能使 OLS 估计偏向零。

\begin{table}[H]
\centering
\caption{IV 有效性诊断}
\begin{tabular}{ll}
\toprule
诊断 & 数值 \\
\midrule
第一阶段 $F$-统计量 & 12.34 \\
第一阶段 $R^2$ & 0.930 \\
弱工具变量检验 & 通过 ($F > 10$) \\
排他性限制 & 理论上合理 \\
偏差修正 (相对于朴素) & 24.5\% \\
\bottomrule
\end{tabular}
\end{table}

第一阶段 $F$-统计量为 12.34,超过了传统的阈值 10 \citep{staiger1997},支持工具变量的相关性。排他性限制在理论上是合理的:历史 ICT 能力仅通过当前的数字能力影响当前排放,前提是控制了我们广泛的控制变量集。

\subsubsection{安慰剂检验}

在100次迭代中置换处理分配产生的 CATE 标准差为 0.041,而真实 CATE 标准差为 0.952,意味着信噪比约为 23:1。真实 ATE 远在安慰剂运行分布之外(伪 $p < 0.001$),提供了强有力的证据表明估计的异质性反映了真实的变异而非统计噪声。

\subsubsection{LOCO 稳定性}

留一国家分析确认了稳健性:全局 ATE 在每个折叠中都保持统计显著,估计值范围为 $-2.33$ 至 $-0.67$。没有任何单一国家主导总体结果。

\subsection{政策例外}

虽然 DCI 显示出一致的负向国家平均效应,但在量级上存在有意义的变异。表~\ref{tab:exceptions} 报告了估计减排最弱的国家。这些国家集中在高收入、高 EDS 经济体——芬兰、瑞典、瑞士和加拿大——这与收益递减和对外专业化假设一致。越南是该组中唯一的非高收入国家,可能反映了与其快速但近期的数字扩张相关的测量问题。

\begin{table}[H]
\centering
\caption{政策例外(减排最弱)}
\label{tab:exceptions}
\begin{tabular}{lccl}
\toprule
国家 & 森林 CATE (DCI) & 95\% CI & 判定 \\
\midrule
FIN & $-0.19$ & $[-0.39, -0.07]$ & 最弱减排 \\
SWE & $-0.46$ & $[-0.60, -0.30]$ & 弱减排 \\
CHE & $-0.50$ & $[-0.57, -0.44]$ & 弱减排 \\
CAN & $-0.52$ & $[-0.62, -0.44]$ & 弱减排 \\
VNM & $-0.90$ & $[-1.01, -0.82]$ & 中度减排 \\
\bottomrule
\end{tabular}
\end{table}

\subsection{异质性来源}

为了理解什么驱动了跨国处理效应的变异,表~\ref{tab:heterogeneity} 报告了国家级 CATE 与关键调节变量之间的相关性。

\begin{table}[H]
\centering
\caption{CATE 与调节变量的相关性}
\label{tab:heterogeneity}
\begin{tabular}{lcl}
\toprule
调节变量 & 相关性 ($r$) & 解读 \\
\midrule
人均 GDP (log) & $-0.33$ & 更高 GDP $\to$ 更强减排 \\
人均能源使用 & $-0.64$ & 最强预测因子 \\
腐败控制 & $-0.09$ & 弱正向一致性 \\
可再生能源 \% & $+0.56$ & 更高可再生能源 $\to$ 更弱减排 \\
\bottomrule
\end{tabular}
\end{table}

人均能源使用均成为处理效应量级的最强预测因子 ($r = -0.64$):人均能源消耗较高的国家经历了更大的 DCI 驱动的减排,这与数字优化在能源浪费更大的地方产生更高回报的逻辑一致。与可再生能源份额的正相关 ($r = +0.56$) 证实了收益递减假说——在已经清洁的电网中,数字效率收益转化为较小的边际碳减排。


\subsection{可视化鸿沟}

图~\ref{fig:linear_vs_forest} 比较了线性交互模型与灵活的因果森林估计。森林检测到了线性模型平滑掉的非线性阈值效应,最大的分歧出现在“甜蜜点”所在的中等收入范围内。

\begin{figure}[H]
\centering
\includegraphics[width=0.85\linewidth]{results/figures/linear_vs_forest.png}
\caption{DCI 效应对人均 GDP 的线性与因果森林估计比较。虚线代表线性交互模型;实线代表因果森林。森林检测到了线性模型遗漏的非线性阈值。}
\label{fig:linear_vs_forest}
\end{figure}

图~\ref{fig:off_diagonal} 识别了“政策例外”——森林预测与线性预测显著偏离的国家。最弱的减排集中在高收入、高 EDS 经济体。

\begin{figure}[H]
\centering
\includegraphics[width=0.85\linewidth]{results/figures/off_diagonal_cis.png}
\caption{非对角线分析:森林预测偏离线性预测的国家。最弱的减排集中在一小部分高收入国家。}
\label{fig:off_diagonal}
\end{figure}

图~\ref{fig:gate} 展示了带有聚类自助法置信区间的 GATE 估计,说明了跨收入组别的倒 U 型模式。

\begin{figure}[H]
\centering
\includegraphics[width=0.85\linewidth]{results/figures/gate_plot.png}
\caption{具有 95\% 聚类自助法置信区间的组平均处理效应。低收入环境下效应适度负向,中等收入环境下强负向,高收入环境下收益递减。}
\label{fig:gate}
\end{figure}

图~\ref{fig:mechanism} 说明了可再生能源份额与 DCI 效应之间的非线性关系,显示随着可再生份额增加,DCI 的碳减排效应减弱。

\begin{figure}[H]
\centering
\includegraphics[width=0.85\linewidth]{results/figures/mechanism_renewable_curve.png}
\caption{机制分析:可再生能源悖论。随着可再生份额增加,DCI 的碳减排效应减弱(趋近于零),支持数字效率在清洁电网中节省较少碳的假设。}
\label{fig:mechanism}
\end{figure}


%==============================================================================
\section{敏感性分析结果}
%==============================================================================

\subsection{自助法收敛诊断}

我们通过改变自助法迭代次数来评估推断的稳定性。表~\ref{tab:bootstrap_conv} 报告了 $B = 100, 200, 500, 1000$ 时的 ATE 估计和置信区间宽度。

\begin{table}[H]
\centering
\caption{自助法收敛诊断}
\label{tab:bootstrap_conv}
\begin{tabular}{lcccc}
\toprule
自助法迭代 & ATE 均值 & SE & 95\% CI 下限 & 95\% CI 上限 \\
\midrule
$B = 100$ & $-1.337$ & 0.114 & $-1.533$ & $-1.138$ \\
$B = 200$ & $-1.333$ & 0.121 & $-1.543$ & $-1.106$ \\
$B = 500$ & $-1.334$ & 0.123 & $-1.550$ & $-1.096$ \\
$B = 1000$ & $-1.332$ & 0.125 & $-1.557$ & $-1.092$ \\
\bottomrule
\end{tabular}
\end{table}

ATE 点估计在自助法迭代中保持稳定(范围从 $-1.337$ 到 $-1.332$),表明点估计已经收敛。然而,置信区间随着 $B$ 的增加略有加宽(从 $B = 100$ 时的宽度 0.395 到 $B = 1000$ 时的 0.465),表明较低的自助法计数可能低估了不确定性。在 $B = 1000$ 时,区间显得稳定,支持我们选择该规格作为主要结果。

\subsection{样本量敏感性}

我们通过对全样本的 60\%、70\%、80\%、90\% 和 100\% 进行子采样来评估结果稳定性。表~\ref{tab:sample_sens} 报告了结果。

\begin{table}[H]
\centering
\caption{样本量敏感性分析}
\label{tab:sample_sens}
\begin{tabular}{lccccc}
\toprule
样本比例 & 国家数 & 观测值 & ATE 均值 & ATE 标准差 & \% 显著 \\
\midrule
60\% & 24 & 504 & $-1.028$ & 1.434 & 45.8\% \\
70\% & 28 & 588 & $-1.516$ & 1.192 & 81.6\% \\
80\% & 32 & 672 & $-0.905$ & 1.607 & 45.8\% \\
90\% & 36 & 756 & $-1.694$ & 1.266 & 68.8\% \\
100\% & 40 & 840 & $-1.316$ & 1.075 & 59.9\% \\
\bottomrule
\end{tabular}
\end{table}

效应在所有子样本中保持方向一致——在每个规格中 ATE 均为负。然而,量级和显著率变化显着:70\% 子样本产生了最高的显著率 (81.6\%),可能反映了构成效应而非统一的精度增益。这些结果强调了在对方向性发现抱有信心的同时,需谨慎解释具体点估计的重要性。

\subsection{动态效应分析}

我们通过估计提前零期到三期的效应来检查 DCI 效应是否随时间持续。表~\ref{tab:dynamic} 报告了结果。

\begin{table}[H]
\centering
\caption{DCI 对 CO$_2$ 排放的动态效应}
\label{tab:dynamic}
\begin{tabular}{lccccc}
\toprule
提前期 & $N$ & ATE & 95\% CI 下限 & 95\% CI 上限 & 显著 \\
\midrule
$t$ (同期) & 840 & $-1.871$ & $-2.903$ & $-0.840$ & 是 \\
$t+1$ & 800 & $-1.566$ & $-2.863$ & $-0.270$ & 是 \\
$t+2$ & 760 & $-1.352$ & $-2.617$ & $-0.088$ & 是 \\
$t+3$ & 720 & $-1.598$ & $-2.772$ & $-0.425$ & 是 \\
\bottomrule
\end{tabular}
\end{table}

效应在两到三年内持续存在,到 $t+3$ 时累计影响达到人均 $-6.39$ 吨。这一发现表明数字基础设施投资带来的是持续而非暂时的气候效益。这种持久性与 DCI 代表结构性能力变化而非一次性调整的假设一致。

\subsection{中介分析}

我们测试能源效率是否中介了 DCI 与排放之间的关系。表~\ref{tab:mediation} 报告了结果。

\begin{table}[H]
\centering
\caption{中介分析结果}
\label{tab:mediation}
\begin{tabular}{lccc}
\toprule
中介变量 & 间接效应 & Sobel $p$值 & 中介比例 \\
\midrule
能源效率 & $-0.343$ & $<0.001$ & 11.7\% \\
结构变迁 & $+0.279$ & $<0.001$ & $-9.5\%$ \\
技术创新 & $+0.245$ & $<0.001$ & $-8.3\%$ \\
\bottomrule
\end{tabular}
\end{table}

能源效率中介了 DCI 对排放总效应的 11.7\% (Sobel 检验 $p < 0.001$),表明 DCI 减少碳产出的途径之一是促进更高效的能源使用。结构变迁和创新的负中介比例表明,这些机制通过不同甚至可能抵消的路径运作:向服务业的结构变迁可能通过构成效应减少排放,但此处捕捉到的间接渠道也反映了增加经济活动等反向力量。


%==============================================================================
\section{讨论}
%==============================================================================

\subsection{数字脱碳鸿沟}

我们的结果揭示了数字化气候影响中依赖于社会经济能力的基础异质性。这种“数字脱碳鸿沟”体现在三个维度上。首先是\textbf{带有显著例外的差距}:富裕国家通常从国内数字能力中获益更多,但一小部分高收入国家——芬兰、瑞典、瑞士和加拿大——显示出明显较弱的减排,这是由其高 EDS 和已清洁的能源系统驱动的。第二个维度是\textbf{EDS 对齐}:较高的对外数字专业化与较弱的减排相关,表明出口结构可能会抑制国内效率收益而不会完全逆转它。第三个维度是\textbf{能源结构鸿沟}:违反直觉的是,可再生能源份额\textit{较低}的国家看到了更强的 DCI 驱动减排。这支持了“边际减排成本”逻辑,即数字优化在能源浪费更大的地方产生更高的碳回报,从而效率提升转化为更大的绝对减排。

\subsection{机制解释}

我们提出两个非互斥的机制来解释观察到的模式。根据\textbf{赋能条件假说},强有力的制度使有效的环境监管成为可能,确保 ICT 带来的效率收益转化为减排而非反弹效应。我们的中介分析支持这一解释:DCI 对排放效应的 11.7\% 是通过提高能源效率运作的 (Sobel 检验 $p < 0.001$)。根据\textbf{结构转型假说},富裕经济体的 ICT 发展代表了向以服务为基础、知识密集型生产的转变,这种生产本质上碳密集度较低。CATE 与可再生能源份额之间的正相关 ($r = +0.56$) 支持\textbf{政策互补性}解释:数字能力和清洁能源基础设施在减排中起到部分替代作用,因此当另一个已经发展良好时,其中一个的边际贡献较小。

我们的三重交互分析进一步揭示,制度调节本身取决于可再生能源份额 ($p < 0.001$)。在可再生能源份额高的国家,制度质量的重要性较低,因为能源系统已经清洁。相反,在依赖化石燃料的国家,强大的制度对于确保数字效率收益转化为实际减排至关重要。

\subsection{政策启示}

对发达经济体而言,总体趋势表明 DCI 可以作为脱碳杠杆。然而,高 EDS 结构性例外表明,在服务出口密集型背景下,效率收益可能较弱。这些环境下的政策应以针对\textit{绝对}脱钩——减少总排放——的措施来补充数字投资,而不仅仅是提高相对效率。碳定价、约束性排放上限和能源系统转型仍然是数字化的重要补充。

对发展中经济体而言,单纯的数字化转型在低能力环境下无法驱动脱碳。在制度能力、电网基础设施和人力资本方面的互补投资是必不可少的先决条件。“甜蜜点”发现表明,国际发展援助应针对拥有数字能力并可利用现有制度基础的中等收入国家,以实现最大的气候回报。

对国际组织而言,这些发现意味着数字发展援助应嵌入更广泛的能力建设方案中,而非孤立部署。优先考虑处于中等收入甜蜜点且制度质量较强的国家,可以最大化每一美元数字投资的碳回报。本文开发的数字 EKC 框架可为特定国家的干预策略提供信息。

\subsection{局限性}

几个局限性值得承认。首先,我们的分析使用了40个国家(840个国家-年份观测)。虽然这些经济体约占全球 GDP 和排放的90\%,但对于因果森林估计而言,40代表了相对较少的独立聚类 \citep{Cameron2008}。读者应将结果解释为提示性证据而非确定性发现;我们报告了 Anderson-Rubin 弱 IV 稳健置信区间以部分解决这一担忧。

其次,虽然 DCI 捕捉了基于基础设施的国内数字能力,但可能无法完全反映数字利用的质量,如 AI 采用强度或数据中心能效。我们的 PCA 诊断显示第一主成分解释了约70\%的方差,留下了有意义的变异未被捕捉。

第三,尽管使用了 DML 框架和 IV 策略,未观测到的混淆因素可能依然存在。我们滞后 DCI 工具变量的排他性限制是理论驱动的,但无法进行实证验证。

第四,我们的主要分析关注同期效应。虽然动态分析表明效应在两到三年内持续存在,但对长期动态——包括潜在的长期反弹效应——的更全面研究将加强这些发现。

第五,样本仅限于主要经济体,发现可能无法推广到未包含在样本中的较小经济体、岛国或最不发达国家。扩展到这些环境是未来研究的重要方向。


%==============================================================================
\section{结论}
%==============================================================================

本文引入了“数字脱碳鸿沟”概念,并为其存在提供了严谨的实证证据。利用因果森林 DML 对40个经济体($N = 840$)的面板进行分析,我们确立了四个主要发现。

首先,国内数字能力 (DCI) 对 CO$_2$ 排放表现出根本性的非线性效应,数字化与脱碳之间的关系随收入分布急剧变化。其次,组平均处理效应揭示了从低能力经济体的中度效应(人均 $-1.19$ 吨)到中等收入经济体的强减排(人均 $-2.17$ 至 $-2.29$ 吨,构成“甜蜜点”),再到高收入经济体收益递减(人均 $-1.26$ 吨)的清晰进程。第三,存在结构性例外:“非对角线”案例证明,高对外数字专业化可以抑制国内效率收益,解释了为什么一些富有、数字先进的经济体显示出令人惊讶的微弱减排。第四,模型阶梯比较表明,灵活估计是捕捉这些政策相关阈值和例外的必要条件,而线性模型系统性地平滑了这些特征。

我们的发现挑战了数字转型对气候目标普遍有益的假设。相反,我们识别了条件先决条件——社会经济能力和制度质量——它们调节了 ICT 是否能带来“绿色红利”。对政策制定者而言,信息是明确的:数字发展援助应优先考虑具有强大制度能力的中等收入国家,在那里每一美元的数字投资都能产生最高的碳回报。


%==============================================================================
\section*{政策工具包附录}
%==============================================================================

\subsection*{政策制定者的决策框架}

基于我们的发现,我们提出了数字气候政策的决策框架,总结在表~\ref{tab:policy_recs} 中。

\begin{table}[H]
\centering
\caption{按国家类型划分的政策建议}
\label{tab:policy_recs}
\begin{tabularx}{\linewidth}{>{\raggedright\arraybackslash}p{3cm} >{\raggedright\arraybackslash}X >{\raggedright\arraybackslash}X}
\toprule
国家类型 & 预期 DCI 影响 & 推荐政策组合 \\
\midrule
低收入 & 中度 ($-1.19$) & 在大规模数字推进前建立互补能力(制度、电网) \\
中低收入 & 强 ($-2.17$) & 优先考虑数字基础设施——最高碳回报 \\
中高收入 & 强 ($-2.29$) & 优先考虑数字基础设施——最高碳回报 \\
高收入 (低 EDS) & 中度 ($-1.26$) & 关注绝对脱钩,不仅仅是效率 \\
高收入 (高 EDS) & 弱 ($-0.19$ 至 $-0.52$) & 解决结构性约束;辅以贸易政策 \\
\bottomrule
\end{tabularx}
\end{table}

\subsection*{实施指南}

对于实施数字气候政策的实践者,我们建议一个五步流程。第一步是通过使用 PCA 框架(互联网用户、宽带订阅、安全服务器)测量现有的 DCI 来评估当前能力。第二步是使用 WGI 指标评估制度质量,以确定有效的数字气候政策的赋能条件是否到位。第三步是通过映射可再生能源份额和能源效率潜力来识别互补性,这共同决定了数字投资的边际减排价值。第四步是锁定甜蜜点,优先考虑具有强大制度的中等收入国家的数字投资,我们的估计表明那里有最高的碳回报。第五步也是最后一步,是通过跟踪效率收益是否转化为绝对减排而非被增加的消费所抵消,来监控反弹效应。


%==============================================================================
\section*{在线补充材料}
%==============================================================================

额外材料可在在线补充附录中获得。附录 A 提供详细的变量定义和数据来源。附录 B 报告完整的 PCA 诊断和组件载荷。附录 C 展示所有规格的完整模型阶梯结果。附录 D 包含带有置信区间的国家级 CATE 估计。附录 E 包括额外的稳健性检查(替代工具变量、安慰剂检验)。附录 F 详述功效分析和蒙特卡洛模拟。附录 G 提供复现代码和数据构建脚本。

复现包(包括代码和数据构建脚本)可在以下网址获取:\url{https://github.com/a985783/digital-decarbonization-divide.git}


%==============================================================================
\section*{声明}
%==============================================================================

\textbf{资助}:本研究未获得公共、商业或非营利部门资助机构的任何具体资助。

\textbf{利益冲突}:作者声明没有利益冲突。

\textbf{数据可用性}:所有原始输入均来自世界银行 WDI/WGI 数据库。处理后的数据和复现代码可在在线补充材料中获得。


%==============================================================================
\bibliographystyle{apalike}
\bibliography{references}

\end{document}
